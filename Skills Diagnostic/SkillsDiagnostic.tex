\documentclass[12pt,letterpaper,boxed,cm]{hmcpset}

% set 1-inch margins in the document
\usepackage[margin=1in]{geometry}
\usepackage{mathtools}
\usepackage{mathrsfs}
% include this if you want to import graphics files with /includegraphics
\usepackage{graphicx}
\usepackage{cases}
\usepackage{hyperref}
\usepackage{siunitx}
\usepackage{tikz}

\name{John Gaskin}
\class{Physics 84}
\assignment{Skills Diagnostic}
\duedate{1/26/17}

\newcommand{\pn}[1]{\left( #1 \right)}
\newcommand{\abs}[1]{\left| #1 \right|}
\newcommand{\bk}[1]{\left[ #1 \right]}
\newcommand{\norm}[1]{\left|\left| #1\right|\right|}
\newcommand{\bra}[1]{\big\langle #1\big\rvert}
\newcommand{\Bra}[1]{\Big\langle #1\Big\rvert}
\newcommand{\ket}[1]{\big\lvert #1\big\rangle}
\newcommand{\Ket}[1]{\Big\lvert #1\big\rangle}
\newcommand{\braket}[2]{\big\langle #1\big\vert #2\big\rangle}
\newcommand{\matelem}[3]{\big\langle #1\big\vert #2\big\vert #3\big\rangle}

\begin{document}
\problemlist{1, 2, 3, 4}

\begin{problem}[1]
    \textbf{Eigenvalues and Eigenvectors}\\
    Find the eigenvalues and eigenvectors of the following matrix: 
    \[
        \begin{bmatrix} 2 & i \\ -i & 2 \end{bmatrix}.
    \]
\end{problem}

\begin{solution}
    \vfill
\end{solution}
\newpage

\begin{problem}[2]
    \textbf{Digital Logic Circuits}\\
    Design and draw a simple logic circuit that takes two classical bits as input and is described by the following truth table:
    \begin{center}
        \begin{tabular}{l c || r}
            A & B & Out \\
            \hline
            0 & 0 & 1 \\
            0 & 1 & 1 \\
            1 & 0 & 0 \\
            1 & 1 & 1 
        \end{tabular}
    \end{center}
\end{problem}

\begin{solution}
    \vfill
\end{solution}
\newpage

\begin{problem}[3]
    \textbf{Qubit States and Measurements}\\
    A qubit is prepared in the quantum state $\ket{\psi} = \frac{1}{\sqrt{3}}\ket{0} + i\sqrt{\frac{2}{3}}\ket{1}$.
    \begin{enumerate}
        \item What is the probability that an ideal projective measurement in the $\{\ket{a},\ket{a_{\perp}}\}$ basis will find the qubit in the state $\ket{a}$ if $\ket{a} = \ket{0}$?
        \item What if $\ket{a} = \frac{1}{\sqrt{2}}(\ket{0}+\ket{1})$?
    \end{enumerate}
\end{problem}

\begin{solution}
    \vfill
\end{solution}
\newpage

\begin{problem}[4]
    \textbf{Matrix Operations}\\
    Write down a matrix $U_{\chi}$ such that $U_{\chi} \begin{bmatrix} a \\ be^{i\varphi} \end{bmatrix} = \begin{bmatrix} a \\ be^{i(\varphi+\chi)} \end{bmatrix}$ for any real numbers $a$, $b$, and $\varphi$.  Is $U_{\chi}$ a unitary matrix?
\end{problem}

\begin{solution}
    \vfill
\end{solution}

% Add pairs of problems and solutions as needed

\end{document}